\documentclass{standalone}
\usepackage{tikz,lmodern,amssymb}
\usepackage{knowledge}

\begin{document}

\begin{defn*}{Field}
A field is a set $\mathbb{F}$, containing at least two elements on which two operations $+$ (addition) and $\cdot$ (multiplication) are defined so that for each pair of elements $x, y \in \mathbb{F}$ there are unique elements $x  +  y$ and $x  \cdot y$ in $\mathbb{F}$ for which the following conditions hold for all elements $x, y , z \in \mathbb{F}$

\begin{itemize}
  \item $x  +  y = y  +  x$ (commutativity of addition)
  \item $\left(  x  +  y  \right)   +  z= x  +  \left( y  +  z \right)$ (associativity of addition)
  \item There is an element $0 \in \mathbb{F}$, which is named $0$ such that $x  +  0 = x$ (existance of additive identity)
  \item For each $x$, we have an element $-x \in \mathbb{F}$ such that $x  +  \left( -x \right) = 0$ (existance of additive inverse)
  \item $xy = yx$ (commutivity of multiplication)
  \item $\left( x  \cdot y \right)  \cdot z = x  \cdot z  +  y  \cdot z$ and $x  \cdot  \left(  y  +  z \right)= x  \cdot y  + x  \cdot z$ (distributivity)
  \item There is an element $1 \in \mathbb{F}$, such that $1 \neq 0$ and $x  \cdot 1 =x$ (existance of a multiplicative identity)
  \item if $x \neq 0$, then there is an element $x ^{-1} \in \mathbb{F}$ such that $x  \cdot  x ^{-1} = 1$ (existance of multiplicative inverses)
\end{itemize}

\end{defn*}

\end{document}



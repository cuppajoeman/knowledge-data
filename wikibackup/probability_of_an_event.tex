Let \(S\) be a sample space, \(E \subseteq S\) be any event and a
function \(P\) satisfying the following properties

\begin{itemize}
\tightlist
\item
  \(P(E) \in [0,1]\)
\item
  \(P(S) = 1\)
\item
  For disjoint events \(E_1, E_2, E_3, \dots\) P\textbackslash{}left(
  \textbackslash{}bigcup\_\{i=1\}\^{}\{\textbackslash{}infty\}
  E\_\{i\}\textbackslash{}right) =
  \textbackslash{}sum\_\{i=1\}\^{}\{\textbackslash{}infty\}
  P\textbackslash{}left(E\_\{i\}\textbackslash{}right)
\end{itemize}

Then we denote \(P(E)\) as the probability of the event \(E\)

\hypertarget{remarks}{%
\subsection{Remarks}\label{remarks}}

\begin{itemize}
\tightlist
\item
  Set \(E_{1}= S\) and for every
  \(i \in \mathbb{N} ^{ \ge 2}, E_{i} = \varnothing\), and note that any
  set and the empty set are disjoint, therefore by the third axiom one
  sees that
  \(P\left(S\right)= \sum_{i=1}^{\infty} P\left(E_{i}\right) = P\left(S\right)  +  \sum_{i=2}^{\infty} P\left( \varnothing\right)\)Thus
  \(P\left( \varnothing\right) = 0\)
\item
  Following that, the third axiom allows us to talk about a finite
  number of disjoint sets
  \(E_{1} , E_{2} , \dotsc  , E_{n - 1} , E_{n}\) as well, to do so, set
  each each \(E_{i} = \varnothing\) for \(i > n\) to get
  \(P\left( \bigcup_{i=1}^{n} E_{i}\right) = P\left( \bigcup_{i=1}^{\infty} E_{i}\right)= \sum_{i=1}^{n} P\left(E_{i}\right)  +  \sum_{i=n  +  1}^{\infty} P\left( \varnothing\right) = \sum_{i=1}^{n} P\left(E_{i}\right)\)
\end{itemize}

\hypertarget{knowledge-used}{%
\subsection{Knowledge Used}\label{knowledge-used}}

\begin{itemize}
\tightlist
\item
  \href{Infinite_Union_of_Sets}{Infinite Union of Sets}
\item
  \url{Function}
\item
  \href{Disjoint_Sets}{Disjoint Sets}
\end{itemize}

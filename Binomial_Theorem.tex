\documentclass[preview]{standalone}
\usepackage{tikz,lmodern,amssymb}
\usepackage{knowledge}

\begin{document}

\begin{theo*}{Binomial Theorem}
Let $x, y \in \mathbb{R}$ such that $x  +  y \neq 0$ then for any $n \in \mathbb{N}$                                                                      
\[                                                                                                                                                        
\left( x + y \right)^{n} = \sum_{i=0}^{n} \binom{n}{i}x^{n-i}y^{i}                                                                                        
\]                                                                                                                                                        
                                                                                                                                                          
\begin{pf}
Consider the product                                                                                                                                      
\[
\left( x_{1}  +  y_{1} \right) \left( x_{2}  +  y_{2} \right) \ldots \left( x_{n}  +  y_{n} \right)
\]
                                                                                                                                                          
The result is a sum of \(2^{n}\) terms wherein, each term is the product of \(n\) factors. Each of these terms will contain as a factor                   
either \(x_{i}\) or \(y_{i}\) for each \(i \in \left[ n \right]\) for                                                                                     example:                                                                                                                                                  
\[                                                                                                                                                        
\left( x_{1}  +  y_{1} \right)\left( x_{2} +  y_{2} \right) = x_{1}x_{2}  +  x_{1}y_{2}  +  x_{2}y_{1}  +  y_{1}y_{2}                                     
\]                                                                                                                                                        
                                                                                                                                                          
For a given term, we will have \(n\) factors, if \(k\) of them are \(x_{i}\)'s then \(n-k\) of the rest are \(y_{i}\)'s. The number of different ways we can choose \(k\) \(x_{i}\)'s from the set \(\left\{ x_{1}, x_{2}, \ldots, x_{n} \right\}\) is simply \(\binom{n}{k}\) and so this corresponds precisely to     the number of terms with \(k\) \(x_{i}\)'s. Thus by letting \(x_{i} = x\), \(y_{i} = y\) for all \(i \in \left\{ 1, \ldots, n \right\}\), we have
\[
  \left( x  +  y \right)^{n} = \sum_{k=0}^{n} \binom{n}{k} x^{k}y^{n-k}
\]
\end{pf}
\end{theo*}

\end{document}



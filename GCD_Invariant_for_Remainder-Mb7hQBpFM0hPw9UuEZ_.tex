\documentclass{standalone}
\usepackage{tikz,lmodern,amssymb}
\usepackage{knowledge}

\begin{document}

\begin{deduction*}{GCD Invariant for Remainder}
  Let $ m,nn \in \mathbb{N} ^{\ge 1}$ such that $ m > n$ and let $ q$ and $ r$ be the unique integers from the quotient remainder theorem, that is, they satisfy
  \[
    m =  q  \cdot n  +  r \qquad \text{ and } \qquad 0 \le r < n
  \]
  Then if $ r > 0$ we have:
  \[
  \gcd(m,n) = \gcd(n,r)
  \]
  or if $ r = 0$, then $ n \mid m$ and $ \gcd(m,n)= n$ 
  \begin{pf}
    \begin{itemize}
      \item If $ r > 0$ 
        \begin{itemize}
          \item Then the equivalent formulation:
          \[
          m - q  \cdot n =  r
          \]
          shows us that if $ d \mid m$ and $ d \mid n$ then $ d \mid r$
        \item Going the other direction would be assuming that $ d$ is a divisor of $ n$ and $ r$, and showing that $ d$ also divides $ m$. This is clear from the original equation:
          \[
            m = q  \cdot  n  +  r
          \]
        \end{itemize}
        \item If $ r =  0$ then we know that $ m = q  \cdot n$ which is the definition of $ n \mid m$ and it's clear that $ \gcd(m, n) = \gcd(q  \cdot n, n) =  n$ 
    \end{itemize}
  \end{pf}
\end{deduction*}

\end{document}



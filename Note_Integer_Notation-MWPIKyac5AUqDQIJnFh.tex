\documentclass{standalone}
\usepackage{tikz,lmodern,amssymb}
\usepackage{knowledge}

\begin{document}

\begin{defn*}{Note Integer Notation}
  \begin{itemize}
   \item Is a notational system which maps the letter names for notes in the western system of music to an integer: 
\[ 
  \huge
\begin{array}{ccccccccccccc}
    C& \cdot& D& \cdot& E& F& \cdot& G& \cdot& A& \cdot& B & C \\
    \updownarrow & \updownarrow & \updownarrow & \updownarrow & \updownarrow & \updownarrow & \updownarrow & \updownarrow & \updownarrow & \updownarrow & \updownarrow & \updownarrow & \updownarrow \\
    \widehat{0} & \widehat{1} & \widehat{2} & \widehat{3} & \widehat{4} & \widehat{5} & \widehat{6} & \widehat{7} & \widehat{8} & \widehat{9} & \widehat{10} & \widehat{11} & \widehat{0}\\
\end{array} 
\] 
    \item The hat is added to denote that we are talking about the pitch produced by playing this note on a device which creates sound.
    \item We may also denote which octave band we are within by writing 
      \[
      \widehat{9}_{4}
      \]
      Which represents an $A4$, the sound generated with a frequency of $440\text{Hz}$ 
    \item We may consider elements such as $ \widehat{12},  \widehat{-1}$ by moving circularly, so that $ \widehat{12} \leftrightarrow C$ and $ \widehat{-1} \leftrightarrow B$. But you can refer to any note using the elements in the initial mapping, so it is standard to use those numbers instead. 
      \begin{itemize}
        \item In other words, without considering which octave a note is in, we have the following equivalence for any $k \in \mathbb{Z}$ and $x \in \left\{ 0, \ldots, 12 \right\}$ 
        \[
        \widehat{x} = \widehat{x  +  12  \cdot k}
        \]
        Which says if you add 12 semitones to any note, it will be the same note differing by an integer number of octaves
      \end{itemize}
  \end{itemize}
\end{defn*}

\end{document}

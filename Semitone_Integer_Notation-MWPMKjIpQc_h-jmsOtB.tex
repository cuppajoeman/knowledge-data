\documentclass{standalone}
\usepackage{tikz,lmodern,amssymb}
\usepackage{knowledge}

\begin{document}

\begin{defn*}{Semitone Integer Notation}
  \begin{itemize}
    \item Is a notational system so that for any two notes $\widehat{x}, \widehat{y} \in \mathbb{W}$ (written in Note Integer Notation), we denote the number of semitones between the two notes as an integer. 
      \begin{itemize}
        \item So instead of saying, $\widehat{y}$ is a perfect 5th above $\widehat{x}$, we would say $\widehat{y}$ is seven above $\widehat{x}$, and write $\widehat{x}  +  7 = \widehat{y}$ 
      \end{itemize}
  \item Since $\widehat{x}, \widehat{y}$ are written in NIN, we have that for any $\alpha \in \mathbb{Z}$ that the note $\widehat{x}  +  \alpha = \widehat{\left( x  +  \alpha \right)}$ 
    \item In general the interval which must be added to $\widehat{x}$ to get to $\widehat{y}$ is $y - x$ 
      \begin{itemize}
        \item From the above it holds: 
          \[
            \widehat{x}  +  \left( y  -  x \right) = \widehat{\left( x  +  y  - x \right)} = \widehat{y}
          \]
      \end{itemize}
  \end{itemize}
    \subsubsection*{Examples}
    \begin{itemize}
      \item If $x = 5$ and $y = 9$, then the interval which must be added is $9 - 5 = 4$ 
      \item It's also possible for it to be a negative number, if $x = 11$ and $y = 2$, then you have to add $2  -  11 = -9$ semitones to $\widehat{x}$ to get to $\widehat{y}$, this corresponds to moving down $9$ semitones
    \end{itemize}

\end{defn*}

\end{document}

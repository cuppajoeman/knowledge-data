\documentclass{standalone}
\usepackage{tikz,lmodern,amssymb}
\usepackage{knowledge}

\begin{document}

\begin{defn*}{Chord Integer Notation}
  Is a notational system for specifying a series of notes to be played simultaneously, written as:
  \[
      \widehat{r} \mid x_{1} , x_{2} , \dotsc  , x_{n - 1} , x_{n}
  \]
  \begin{itemize}
    \item Where $ \widehat{r} \in \mathbb{W}$ is the root tone 
    \item  $ x_{1} , x_{2} , \dotsc  , x_{n - 1} , x_{n}$ are a series of intervals that generate the notes
      \[
      \widehat{r  +  x_{1}}, \widehat{r  +  x_{2}}, \ldots, \widehat{r  +  x_{n}}, 
      \]
      These notes form the chord.
  \end{itemize}
  \subsubsection*{Examples}
  \begin{itemize}
    \item $ \widehat{0} \mid 0, 4, 7, 11$ generates the notes $ \widehat{0}, \widehat{4}, \widehat{7}, \widehat{11} \leftrightarrow C, E, G, B$ which is a C major 7th chord.
    \item $ \widehat{5} \mid 0, 3, 7$ generates the notes $ \widehat{5}, \widehat{8}, \widehat{0} \leftrightarrow F, Ab, C$ which is a F minor triad
  \end{itemize}
\end{defn*}

\end{document}
